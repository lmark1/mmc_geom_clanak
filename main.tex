%%%%%%%%%%%%%%%%%%%%%%%%%%%%%%%%%%%%%%%%%%%%%%%%%%%%%%%%%%%%%%%%%%%%%%%%%%%%%%%%
%2345678901234567890123456789012345678901234567890123456789012345678901234567890
%        1         2         3         4         5         6         7         8

\documentclass[letterpaper, 10 pt, conference]{ieeeconf}  % Comment this line out
                                                          % if you need a4paper
%\documentclass[a4paper, 10pt, conference]{ieeeconf}      % Use this line for a4
                                                          % paper

\IEEEoverridecommandlockouts                              % This command is only
                                                          % needed if you want to
                                                          % use the \thanks command
\overrideIEEEmargins
% See the \addtolength command later in the file to balance the column lengths
% on the last page of the document

\pdfminorversion=4

% The following packages can be found on http:\\www.ctan.org
%\usepackage{graphics} % for pdf, bitmapped graphics files
%\usepackage{epsfig} % for postscript graphics files
%\usepackage{mathptmx} % assumes new font selection scheme installed
%\usepackage{times} % assumes new font selection scheme installed
%\usepackage{amsmath} % assumes amsmath package installed
%\usepackage{amssymb}  % assumes amsmath package installed
\usepackage{amsmath}    			% ams packages for mathematics environment    
\usepackage{amssymb}
\usepackage{amsfonts}
%\usepackage{amsthm}

\usepackage{graphicx}  				% Versatile graphics manipulation options

\usepackage[croatian]{babel}  % Croatian typographical rules and hyphenation patterns 
\usepackage[utf8]{inputenc}  	% Encoding of Croatian characters
\usepackage[T1]{fontenc}
\usepackage{ae,aecompl}     	% Type 1 fonts, similar to Computer Modern

\usepackage{microtype}				% Improves spacing

\usepackage{subfig}

\usepackage{tabularx}
\usepackage{booktabs}
%\newcolumntype{C}{>{\centering\arraybackslash}X} % centered version of "X" type
\setlength{\extrarowheight}{1pt}
\usepackage{enumerate}				% Additional options for listing of items in enumerate environment
\usepackage{algorithm2e}			% Writing pseudo-code
\usepackage{todonotes}				% Adding todo items
\usepackage{dirtree}					% Simple display of directory tree
\usepackage{hyperref}					% Managing cross-referencing

\usepackage{scalerel,stackengine}
\stackMath
\newcommand\reallywidehat[1]{%
	\savestack{\tmpbox}{\stretchto{%
			\scaleto{%
				\scalerel*[\widthof{\ensuremath{#1}}]{\kern.1pt\mathchar"0362\kern.1pt}%
				{\rule{0ex}{\textheight}}%WIDTH-LIMITED CIRCUMFLEX
			}{\textheight}% 
		}{2.4ex}}%
	\stackon[-6.9pt]{#1}{\tmpbox}%
}
\parskip 1ex

%calligraphy packages
\usepackage{calrsfs}
\DeclareMathAlphabet{\pazocal}{OMS}{zplm}{m}{n}
\newcommand{\Ca}{\pazocal{C}}
\newcommand{\Oa}{\pazocal{O}}
\newcommand{\Va}{\pazocal{V}}
\newcommand{\Ua}{\pazocal{U}}
\newcommand{\Aa}{\pazocal{A}}
\newcommand{\Ta}{\pazocal{T}}
\newcommand{\La}{\pazocal{L}}

\newcommand{\Ja}{\pazocal{J}}

\graphicspath{{./figures/}}
\usepackage{float}

\title{\LARGE \bf
Geometric Tracking Control of Aerial Robots Based on Centroid Vectoring
}
\author{Lovro Marković, Antun Ivanović, Marko Car, Matko Orsag, Stjepan Bogdan
	\thanks{Authors are with Faculty of Electrical and Computer Engineering,
        University of Zagreb, 10000 Zagreb, Croatia
        {\tt\small (Antun Ivanović, Marko Car, Tomislav Haus, Matko Orsag, Stjepan Bogdan, Lovro Marković) at fer.hr}}}%

\makeatletter
\newcommand{\removelatexerror}{\let\@latex@error\@gobble}
\newcommand{\mb}[1]{\boldsymbol{#1}}
\makeatother

\begin{document}
\maketitle

\thispagestyle{empty}
\pagestyle{empty}


%%%%%%%%%%%%%%%%%%%%%%%%%%%%%%%%%%%%%%%%%%%%%%%%%%%%%%%%%%%%%%%%%%%%%%%%%%%%%%%%
\begin{abstract}

This paper focuses on presenting the concept of geometric tracking control for an unmanned aerial vehicle (UAV) based on variations in center of gravity (CoG). The proposed UAV model has the ability to exploit its dynamic CoG as a means of stabilization and control. A mathematical model of such a system is used as a base for developing the nonlinear geometric tracking controller on the special Euclidean group SE(3). Finally, two unique UAV models, presented with a trajectory tracking problem, are simulated in a realistic simulation environment. Performance of the selected control terms is analyzed based on relevant simulation results.

\end{abstract}
\section{introduction}
With interest in unmanned aerial vehicles growing rapidly over the course of last decade, aerial manipulation has become one of the most prominent research fields of robotics. Application scenarios include contact based inspection \cite{Alexis2016}, transportation and assembly \cite{Jimenez-Cano2013} and even complex interaction tasks involving opening doors \cite{Tsukagoshi2015} or turning valves \cite{Korpela2014a}. Since the very beginning of aerial manipulation, payload has been observed as a disturbance to the autopilot control system relying on the rotors to stabilize the aircraft. Pounds et al. explored the effect added payload and the corresponding shift in the center of mass have on the stability of a standard linear PID attitude controller \cite{Pounds2012}. In \cite{PalunkoIFAC2011} authors proposed an adaptive control approach to solve the problem of dynamic variations in the center of mass while transporting additional load. In \cite {Korpela2012} we analyzed the variable moment of inertia caused through motion of a multi degree of freedom manipulator. A common denominator of the state-of-the-art is that it observes a payload as a disturbance to the classical control approach of rotorcraft aerial vehicles. 


In this paper we seek to invert this observation and use a controller capable of utilizing centroid variations in order to actively control the attitude of the aerial robot. This in turn enables us to use the payload to help stabilize the Unmanned Aerial System's (UAS) body. As the rotorcraft design increases payload-to-weight ratio, the center of mass becomes ever more dominant dynamic component of the system. The ability to utilize it as a control mechanism, rather than a control disturbance, could potentially help improve overall UAS performance. This is in line with our long term goal within the MORUS project \cite{MORUSweb}, where we aim to build an unmanned aerial vehicle powered with four internal combustion engines (ICE), weighting 30kg in total and capable of lifting a 50kg heavy unmanned underwater vehicle (UUV). In order to solve the inherent problem of slow dynamics of ICE powered rotors, in \cite{Haus2017} we proposed using a Moving Mass Control (MMC) concept. Similar concept was reported in \cite{bermes2008new} used on a coaxial helicopter vehicle.

Herein we will use a scaled model of the MORUS UAS shown in Fig. \ref{fig:FlyingSCARA} and dubbed $\mu$MORUS, to demonstrate the effectiveness of the proposed control concept. Unlike in \cite{bermes2008new}, $\mu$MORUS has four rotors which are used together with moving masses in order to stabilize the vehicle. We combine these two control concepts through a paradigm known as mid-ranging control. This paradigm has been introduced in process industry for plant control with multiple actuators working simultaneously \cite{Allison1997MidRanging},\cite{Allison2003MidRanging}. In recent years, this type of control has gained attention in robotics, in particular for position control of a robotic arm consisted of a macro and mini manipulator \cite{Sornmo2013AdaptiveImc}, \cite{Ma2015MidRanging}. We further extended the results from \cite{Haus2017} by analyzing stability of the mid-ranging control concept, as well as applying such a control scheme on a multirotor UAV \cite{haus2018med}. 

CONTRIBUTIONS!!!!!! Moreover, in this paper we aim to derive a generalized mathematical model that can be applied to a generic aerial robot consisting of a UAV body and a multi degree of freedom manipulator potentially carrying a payload. 

The paper is organized as follows. We start by deriving a complete mathematical model of an aerial robot consisting of rotorcraft UAV body, and a generic multiple degree of freedom manipulator. Next we present the proposed, combined control system based on mid-ranging paradigm, Valve Position Control (VPC), utilizing both classical rotor and moving mass control concepts. Furthermore, we present the simulation results of a dual arm aerial robot with RR  manipulators carrying an object. Finally, we present the experimental results for the $\mu$MORUS UAS. To compare the proposed control paradigm with classical approach single axis of the UAS is equipped with moving masses.
\section{Mathematical model}
\todo[inline]{TODO: Mathematical model...}


\section{Mid-ranging control concept}

\todo[inline]{TODO: Control...}
\section{Simulation}
\label{sec:simulation}
To validate the VPC concept with PID-based position controller on an arbitrary aerial manipulator we created realistic simulation environment in Gazebo simulator within Robot Operating System (ROS). We model the UAV a as single rigid body with rotating joints attached to each of four motor arms. Furthermore, each rotating joint has a propeller attached. In order to simulate rotor dynamics we use a plugin from open source \textit{rotors\_simulator} \cite{Furrer2016} package. We also equip the UAV with realistic sensors from \textit{hector\_gazebo} \cite{2012simpar_meyer} to measure vehicle's attitude and pose.

To control the COG, moving mass is attached to each arm of the UAV with mass $m=0.2kg$, while the central body of the UAV is considered to be a single rigid body. Standard PID control is utilized to move the masses along the arms of the UAV while restricting the workspace to $\Delta x \in [-0.08m, 0.08m]$. The aforementioned MMC-VPC concept is utilized to control attitude and cascade PID controller is used to control position. Two simulation experiments were conducted to validate the performance of the controller: hovering and trajectory following. The RMS measure is used to compare results.

While hovering for $t_h = 180s$, the UAS achieved position score $RMS_h = 0.00076m$. For the trajectory tracking, a square trajectory of side length $a=1m$ was chosen as the example. A set of $10$ identical trajectories were performed by the UAS, each lasting for $t=20.89s$. Figure \ref{fig:simulation_x_trajectory} depicts the trajectory tracking. The average position score over $10$ trajectories was $RMS_{pos} = 0.2946m$. Figure \ref{fig:simulation_pitch_trajectory} depicts pitch angle tracking performance. One can notice a very small delay while achieving great tracking of the reference, which can be backed up with $RMS_{\theta} = 0.0021^\circ$. Similar result is obtained for roll angle with $RMS_{\phi} = 0.0019^\circ$.  

%\section{Experiments}
To demonstrate the proposed MMC-VPC control algorithm in action, we present the results of several experiments conducted on a real physical platform $\mu$MORUS, shown in Fig. \ref{fig:FlyingSCARA}. Our aerial robot is a 3D Robotics quadrotor equipped with four moving masses (NEMA 14 stepper motors) with a rack and pinion mechanism that transform rotational motion into linear motion. Masses are placed on each arm of the $\mu$MORUS platform. At the same time rotors are symmetrically placed around the central body in a pattern known as plus (+) configuration. We use a custom designed printed circuit board to command stepper motors and Pixhawk PX4 as the flight controller. For out first experiment we wanted to compare MMC-VPC control algorithm and classical rotor speed control in stabilization flight. We implemented MMC-VPC for pitch angle control while classical algorithm was controlling the roll angle. The second experiment was conducted to test the controller performance on simple trajectory. For that case, the MMC-VPC algorithm is implemented for both axis together with position controller. All control algorithms are implemented on the Pixhawk PX4 flight controller and the off-board computer is used for data logging. For testing purposes $\mu$MORUS is powered over electric cables and connected to the off-board computer through a USB cable.


%\subsection{Constrained 2DOF motion}

%First experiment was to test and tune control algorithm until satisfactory flight performance was achieved. To do so, we constrained $\mu$MORUS by mounting it on 2DOF gimbal. The sequence of commanded references and the response of vehicle’s pitch angle is presented in Fig. \ref{fig:pitch_kut_klackalica}. The MMC-VPC algorithm output is given in Fig. \ref{fig:polozaj_masa_klackalica}. Measured attitude data shows satisfying command tracking performance with RMS error of $0.038rad$, average overshoot of $27.5\%$ and average rise time $t_{r} = 0.32s$. The noise in moving masses motion is caused by rotor vibrations and unmodeled dynamics. Since we have shown that the moving masses are operating in higher bandwidth, they are responsible for initial, transient response, which one can observe in Fig. \ref{fig:pitch_kut_klackalica}. After certain amount of time, the MMC-VPC controller adjusts rotors' speed and the masses return to the center point of their operating range. This is identical to the results shown and discussed in Section \ref{sec:simulation}, and derived in Section \ref{sec:control}.

%\begin{figure}[h!]
%  \centering
%  \subfloat[]{\includegraphics[width=0.48\textwidth]{./pictures/%pitch_kut_klackalica}\label{fig:pitch_kut_klackalica}}
%  \hfill
%  \subfloat[]{\includegraphics[width=0.48\textwidth]{./pictures/polozaj_masa_klackalica}\label{fig:polozaj_masa_klackalica}}
%  \caption{Experimental results of $\mu$MORUS UAS on gimbal. a) shows a sequence of pitch references and the corresponding responses. b) shows a position setpoint of the moving masses and MMC-VPC control algorithm output for rotors.}
%\end{figure}

\subsection{Manual stabilization flight}

Our first experiment was manual stabilization flight with $\mu$MORUS. Pilot took-off with the vehicle, hovered for few minutes and then landed. Fig. \ref{fig:roll_pitch_kut_let} represents pitch and roll measurements during experiment. Considering attitude measurements in Fig. \ref{fig:roll_pitch_kut_let}, the results show that there is no significant difference between the two control paradigms. Results shows a stable flight, where both roll and pitch angles are within a few degrees. One has to be aware that the pilot was manually trying to keep the vehicle steady during the whole experiment, and that our goal was to show the proposed concept can stabilize the UAS in flight. The reference for MMC-VPC algorithm outputs, rotors 3 and 4, are given in Fig. \ref{fig:rc_out_let} alongside rotor 1 and rotor 2 which are controlled using the standard attitude controller. One can notice that the rotors commanded with MMC-VPC operate in lower bandwith. This is in line with the expected results, since the moving masses control attitude during the transient period which requires faster motion. This results with smother rotors' references, when compared to rotors 1 and 2. Nevertheless, the response of the UAV remains the same, as shown in Fig. \ref{fig:roll_pitch_kut_let}. The parameter used for attitude control loop are given in Table \ref{table:attitude_control_params}.

\begin{table}[h!]
\centering
\caption{MMC-VPC controller gains for attitude control loop where.}
\label{table:attitude_control_params}
\begin{tabular}{|c|c|c|c|c|c|c|}
\hline
 & $\phi$ & $\dot{\phi}$ & $\theta$ & $\dot{\theta}$ & $\psi$ & $\dot{\psi}$\\
\hline
$KP$ & $0$ & $0$ & $0$ & $0$ & $0$ & $0$\\
\hline
$KI$ & $0$ & $0$ & $0$ & $0$ & $0$ & $0$\\
\hline
$KD$ & $0$ & $0$ & $0$ & $0$ & $0$ & $0$\\
\hline
$KI_{VPC}$ & $0$ & $0$ & $0$ & $0$ & $0$ & $0$\\
\hline
\end{tabular}
\end{table}

\begin{figure}
  \centering
  \subfloat[]{\includegraphics[width=0.5\textwidth]{./pictures/pitch_kut_let}\label{fig:roll_pitch_kut_let}}
  \hfill
  \subfloat[]{\includegraphics[width=0.5\textwidth]{./pictures/rc_out_motor_ref_let}\label{fig:rc_out_let}}
  \caption{Experimental results of $\mu$MORUS UAS in stabilization flight. Roll and pitch measurements are depicted in a). b) shows reference for rotors 1-4. MMC-VPC takes higher bandwidth out of rotor 3 and 4 control, which results with smoother rotors' reference when compared to classically controlled rotors 1 and 2.}
\end{figure}

\subsection{Trajectory following}

The second experiment was conduced in order to test MMC-VPC algorithm with higher level control. We implemented position control in the standard cascade control form with PID controller. To tune our position controller we started with parameters from simulation and with some fine tuning we reached parameters shown in Table \ref{table:position_control_params}. 

\begin{table}[h!]
\centering
\caption{PID controller gains for cascade position control where $x$, $y$ and $z$ denotes outer control loop (position) and $vx$, $vy$ and $vz$ denotes inner control loop (velocity).}
\label{table:position_control_params}
\begin{tabular}{|c|c|c|c|c|c|c|}
\hline
 & $x$ & $vx$ & $y$ & $vy$ & $z$ & $vz$\\
\hline
$KP$ & $0$ & $0$ & $0$ & $0$ & $0$ & $0$\\
\hline
$KI$ & $0$ & $0$ & $0$ & $0$ & $0$ & $0$\\
\hline
$KD$ & $0$ & $0$ & $0$ & $0$ & $0$ & $0$\\
\hline
\end{tabular}
\end{table}

We can divide second experiment in to a two parts. The part a) of the experiment was to hover (maintain the constant position) with the $\mu$MORUS UAS and the part b) was to follow trajectory. On Fig. \ref{fig:position_hover} is shown position setpoint and feedback of the UAS while hovering. The UAS was able to maintain constant position with RMS error for position in x-axis $RMS_X = 0.0739 m$ and y-axis $RMS_Y = 0.0424 m$. The low level controller states for roll axis are shown on Fig. \ref{fig:roll_hover}, while pitch axis is almost identical. Here we can point out RMS error for roll angle $RMS_\phi = 2.091^{\circ}$ and RMS error for pitch angle $RMS_\theta = 2.354^{\circ}$.

\begin{figure}[h!]
\centering
\includegraphics[width=\columnwidth]{./pictures/arducopter_y_hover}
\caption{Results of the hovering experiment. Figure shows position setpoint and measurement for y-axis. Similar results are obtained for x-axis.}
\label{fig:position_hover}
\end{figure} 

\begin{figure}[h!]
  \centering
  \subfloat[]{\includegraphics[width=0.5\textwidth]{./pictures/arducopter_roll_hover}\label{fig:roll_hover}}
  \hfill
  \subfloat[]{\includegraphics[width=0.5\textwidth]{./pictures/arducopter_roll_rate_hover}\label{fig:rollrate_hover}}
  \caption{The experimental results of the $\mu$MORUS UAS during hovering experiment. a) shows a setpoints and the corresponding measurements for angle $\phi$. b) shows a setpoints and the corresponding measurements for angular velocity $\dot{\phi}$. The results for $\theta$ angle are similar.}
  \label{fig:roll_hover}
\end{figure}

The b) part of the experiment (following a trajectory) give us a position results that are shown on Fig. \ref{fig:experiment_position}. The UAS was able to follow trajectory with RMS error $RMS = 0.3139 m$. The roll angle and roll rate during the trajectory execution are shown on Fig. \ref{fig:experiment_roll}. The video showing both experiments can be found in \cite{uMORUS2017video}. The Github repository with source code for both simulation can bi found in \cite{letaci2017} and experiments in \cite{letaciPixhawk2017}.

\begin{figure}[h!]
\centering
\includegraphics[width=\columnwidth]{./pictures/arducopter_y}
\caption{Results of the trajectory following. Figure shows position setpoint and measurement for y-axis. Similar results are obtained for x-axis.}
\label{fig:experiment_position}
\end{figure}


\begin{figure}[h!]
  \centering
  \subfloat[]{\includegraphics[width=0.5\textwidth]{./pictures/arducopter_roll}\label{fig:experiment_roll_angle}}
  \hfill
  \subfloat[]{\includegraphics[width=0.5\textwidth]{./pictures/arducopter_roll_rate}\label{fig:experiment_roll_rate}}
  \caption{The experimental results of the $\mu$MORUS UAS during the trajectory following experiment. a) shows a setpoints and the corresponding measurements for angle $\phi$. b) shows a setpoints and the corresponding measurements for angular velocity $\dot{\phi}$. The results for $\theta$ angle are similar.}
  \label{fig:experiment_roll}
\end{figure}
\section{Conclusion}
Geometric control was presented and implemented for two separate UAV models with variable centers of gravity. Although the controller has been constructed without actuator dynamics in mind, during the simulations its effect can be seen. \\
Moving mass actuator dynamics have a noticeably faster transient effect than the manipulator dynamics. This claim can be affirmed by observing \ref{fig:cog_error} where variations in CoG produced by MMC occur quicker and achieve greater magnitudes than those produced by the carried payload. Therefore, it is expected that the UAV with MMC achieves better position and attitude tracking than UAV carrying a payload. In \ref{fig:traj_pos} it can be seen that this expectation is validated. In the first case(MMC) excellent tracking performance can be observed, while in the second case(payload) measured position along the y-axis exhibits tracking delay and lower magnitude with respect to the desired position. \\
The overall effect of control terms \ref{force_control} and \ref{moment_control} which include $\textbf{r}_{CoG}$ becomes negligible if considering slower trajectories. In such cases, both UAVs can be observed as having CoG inside the origin of the body-fixed frame yielding simpler control terms.




%%%%%%%%%%%%%%%%%%%%%%%%%%%%%%%%%%%%%%%%%%%%%%%%%%%%%%%%%%%%%%%%%%%%%%%%%%%%%%%%
\section*{APPENDIX} \label{sec:appendix}
In this section rotating body dynamics with variations in center of gravity will be derived.
General form of Euler-Lagrange dynamics for a rotating rigid body in SE(3) configuration manifold in the body-fixed frame as presented in \cite{LeeModel}:
\begin{gather}
	\frac{d}{dt} \left( \frac{\partial \La}{\partial \mb{\Omega}} \right)
	+ \mb{\Omega} \times \frac{\partial \La}{\partial \mb{\Omega}} 
	+ \textbf{v} \times \frac{\partial \La}{\partial \textbf{v}} 
	+ \sum_{i=1}^{3} \textbf{r}_i \times \frac{\partial \La}{\partial \textbf{r}_i} = 0 \label{general1}\\
	\frac{d}{dt} \left( \frac{\partial \La}{\partial \textbf{v}} \right)
	+ \mb{\Omega} \times \frac{\partial \La}{\partial \textbf{v}} 
	- \text{R}^T \frac{\partial \La}{\partial x} = 0 \label{general2}
\end{gather}

For the the proposed UAS with variations in center of mass the Lagrangian is:
\begin{equation}
	\La(\text{R},x,\mb{\Omega},\textbf{v}) = \frac{1}{2}\mb{\Omega}^TJ\mb{\Omega} + m \mb{\Omega}^T \hat{\textbf{r}}_{cm}\textbf{v} + \frac{1}{2}m\textbf{v}^T\textbf{v} - U(\text{R},\textbf{x})
\end{equation}

\noindent where $U(\text{R}, \textbf{x})$ is the potential energy of the system. It is important to note that $J$ and $\textbf{r}_{cm}$ are variable over time. \\
Lagrangian derivatives needed for the general form equations \ref{general1} and \ref{general2} are:

\begin{gather}
	\frac{\partial \La}{\partial \mb{\Omega}} = J\mb{\Omega} + m \hat{\textbf{r}}_{cm})\textbf{v} \label{d1}\\ 
	\frac{d}{dt} \left( \frac{\partial \La}{\partial \mb{\Omega}} \right) = \dot{J} \mb{\Omega} + J \dot{\mb{\Omega}} + m \dot{\textbf{r}}_{cm} \times \textbf{v} + m \textbf{r}_{cm} \times \dot{\textbf{v}} \label{d2}\\ 
	\frac{\partial \La}{\partial \textbf{v}} = m\textbf{v} - m\textbf{r}_{cm} \times \mb{\Omega} \label{d3}\\ 
	\frac{d}{dt} \left( \frac{\partial \La}{\partial \textbf{v}} \right) = m\dot{\textbf{v}} - m\dot{\textbf{r}}_{cm} \times \mb{\Omega} - m \textbf{r}_{cm} \times \dot{\mb{\Omega}} \label{d4}
\end{gather}

It is of interest to transfer rotation and translation dynamics in the inertial frame. This can be done using the following relations:

\begin{gather}
	\textbf{v} = \text{R}^T \dot{x} \label{inertial1}\\
	\dot{\textbf{v}} = \text{R}^T \ddot{\textbf{x}} - \mb{\Omega} \times (\text{R}^T \dot{\textbf{x}} ) \label{inertial2} \\
	\textbf{r}_{cm} = \text{R}^T(\textbf{x}_{cm} - \textbf{x}) \label{inertial3} \\
	\dot{\textbf{r}}_{cm} = \text{R}^T(\dot{\textbf{x}}_{cm} - \dot{\textbf{x}}) + \text{R}^T\hat{\mb{\Omega}}\textbf{x}_{cm} - \hat{\mb{\Omega}}\text{R}^T(\textbf{x}_{cm} - \textbf{x}) \label{inertial4}
\end{gather}


After plugging in \ref{d1}, \ref{d2}, \ref{d3}, \ref{d4} in \ref{general1}, \ref{general2} and using \ref{inertial1}, \ref{inertial2} as transformations of velocity and acceleration to inertial frame the following equations are obtained:
\begin{align}
	\begin{split}
		& J\dot{\mb{\Omega}} + m \textbf{r}_{cm} \times \text{R}^T \ddot{\textbf{x}} \\
		&+ \dot{J} \mb{\Omega} + m\dot{\textbf{r}}_{cm} \times \text{R}^T \dot{\textbf{x}} \\
		&+ \mb{\Omega} \times J\mb{\Omega} + \sum_{i=1}^{3} \textbf{r}_i \times \frac{\partial \La}{\partial \textbf{r}_i} = 0
	\end{split}
\end{align}
\begin{align}
	\begin{split}
		& m\ddot{\textbf{x}} - m\text{R}\hat{\textbf{r}}_{cm} \dot{\mb{\Omega}} - m\text{R}\hat{\dot{\textbf{r}}}_{cm}\mb{\Omega}\\
		&-m\text{R}\hat{\mb{\Omega}}\hat{\textbf{r}}_{cm}\mb{\Omega} + \frac{\partial U(\text{R},\textbf{x})}{\partial \textbf{x}} = 0
	\end{split}
\end{align}

After plugging in the center of mass transform \ref{inertial3}, \ref{inertial4} the final form of dynamics is obtained:
\begin{align}
	\begin{split}
		 & J\dot{\mb{\Omega}} + m\text{R}^T(\textbf{x}_{cm} - \textbf{x}) \times \text{R}^T\ddot{\textbf{x}} + \dot{J}\mb{\Omega} \\
		 & + m\reallywidehat{\text{R}^T\dot{\textbf{x}}_{cm}}\text{R}^T\dot{\textbf{x}} - \text{R}^T\dot{\textbf{x}} \times (\text{R}^T\hat{\mb{\Omega}}\textbf{x}_{cm}) \\
		 & - m\hat{\mb{\Omega}}\text{R}^T(\textbf{x}_{cm} - \textbf{x}) \times \text{R}^T \dot{\textbf{x}} \\
		 & + \mb{\Omega} \times J\mb{\Omega}  + \sum_{i=1}^{3} \textbf{r}_i \times \frac{\partial \La}{\partial \textbf{r}_i} = 0
	\end{split}
\end{align}
\begin{align}
	\begin{split}
		m \ddot{\textbf{x}} &- m\text{R}\reallywidehat{\text{R}^T(\textbf{x}_{cm} - \textbf{x})}\dot{\mb{\Omega}} \\
		& - m\text{R}\reallywidehat{\text{R}^T(\dot{\textbf{x}}_{cm} - \dot{\textbf{x}})} \mb{\Omega} \\
		& - m\text{R}\reallywidehat{\text{R}^T\hat{\mb{\Omega}}\textbf{x}_{cm}}\mb{\Omega} \\
		& + \frac{\partial U(\text{R},\textbf{x})}{\partial \textbf{x}} = 0
	\end{split}
\end{align}


%Appendixes should appear before the acknowledgment.

\section*{ACKNOWLEDGMENT}

This research was supported in part by NATO's Emerging Security Challenges Division in the framework of the Science for Peace and Security Programme as Multi Year Project under G. A. number 984807, named Unmanned system for maritime security and environmental monitoring - MORUS.


%%%%%%%%%%%%%%%%%%%%%%%%%%%%%%%%%%%%%%%%%%%%%%%%%%%%%%%%%%%%%%%%%%%%%%%%%%%%%%%%

\nocite{*}
\bibliographystyle{ieeetr}
\bibliography{bibliography/Mendeley}

\end{document}
