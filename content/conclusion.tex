\section{Conclusion} \label{sec:conclusion}
Geometric control was presented and implemented for two separate UAV (Unmanned Aerial Vehicle) models with variable centers of gravity. Although the controller has been constructed without actuator dynamics in mind, during the simulations its effect can be seen. \\
Moving mass actuator dynamics have a noticeably faster transient effect than the manipulator dynamics. This claim can be affirmed by observing Figure \ref{fig:cog_error} where variations in CoG (Center of Gravity) produced by MMC (Moving Mass Control) occur quicker and achieve greater magnitudes than those produced by the carried payload. Therefore, it is expected that the UAV with MMC achieves better position and attitude tracking than UAV carrying a payload. In Figure \ref{fig:traj_pos} it can be seen that this expectation is validated. \\
In the first case (MMC) excellent tracking performance can be observed, while in the second case (Manipulator Carried Payload) measured position along the y-axis exhibits tracking delay and lower magnitude w.r.t. the desired position. This is partly due to the given trajectory, whose fast dynamic around y-axis proves unmanageable for the UAV. Moreover, limited movement radius of the carried payload greatly inhibits position tracking at a higher velocity. \\   
The overall effect of control terms \eqref{force_control} and \eqref{moment_control} which include $\textbf{r}_{CoG}$ becomes negligible if considering slower trajectories. In such cases, both UAVs can be observed as having CoG inside the origin of the body-fixed frame due the insignificant disturbance caused by CoG variations. 
