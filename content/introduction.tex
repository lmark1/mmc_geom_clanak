\section{introduction}

Geometric control concept has previously been applied for classic quadrotor vehicles in \cite{LeeClanak4}, \cite{LeeClanak3}, \cite{LeeClanak1} set up in plus configuration with their center of gravity located inside the origin of the UAV body frame. Since a specific type of UAV is considered in this paper, a different approach to modeling such a system will be taken. Unlike a standard UAV, it utilizes variations in its center of gravity in order to achieve attitude tracking. Essentially, this means that such variations, which would usually be considered a disturbance in the system, could be exploited as a means of controlling the UAV. 
One of the ways these variations will be achieved is by implementing the moving mass concept on the standard quadrotor UAV. This includes mounting moving masses on the UAV axes, whose offset will act as the control input of the system along with rotor speed variation. This is a novel concept first developed in \cite{movingMass1},\cite{movingMass2} with attitude control considered in \cite{movingMass3}. Up to this point, the concept of geometric control has not been applied to the moving-mass controlled UAV.  \\
Another way such variations could be achieved is by mounting two manipulators to the UAV and having them carry a relatively heavy payload - compared to the total UAV mass. In this case position of the payload will directly determine any offset in center of gravity. UAVs endowed with manipulators have previously been studied in \cite{manipulator1}, \cite{manipulator2}. Similar trajectory tracking problem has previously been presented in \cite{manipulator3}. In this paper, however, two 3-DOF manipulators are used, each carrying a payload. They are considered as an extension of the UAV used only for attitude control and not the main subject for trajectory tracking. \\
Since the geometric controller is used in this paper, model dynamics need to be expressed on SE(3) configuration manifold. Similar quadrotor dynamics have already been considered in other research papers e.g. \cite{LeeClanak4}, \cite{LeeClanak2}, \cite{LeeClanak1} in which center of gravity lies in the origin of the body-fixed frame. However, in this paper it is not the case. Due to the variations in center of gravity a different dynamic model will be taken in consideration. \\
Therefore, the goal of this paper is to present the appropriate dynamic model for UAVs with variable center of gravity on the SE(3) configuration manifold, choose control terms based on that model and evaluate controller performance on a predefined trajectory tracking problem using two different UAV models able to change their center of gravity. \\
UAV model used in Gazebo simulations will be $\mu$MORUS which is a scaled down version of a UAV used in the MORUS project \cite{MORUSweb}. \\
The paper is organized as follows. First the general mathematical model will be presented on the SE(3) configuration manifold along with expressions for center of gravity and moments of inertia. Using that mathematical model, control terms will be chosen such that desirable error dynamics can be obtained. Sufficient stability conditions will be presented for the obtained error dynamics. Lastly, two sets of simulations will be conducted using the Gazebo simulator and ROS environment. A trajectory tracking problem will be given to both UAV models, results will be presented and compared.