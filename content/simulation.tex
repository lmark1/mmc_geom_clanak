\section{Simulation}
Simulations are conducted in the Gazebo simulator within the ROS environment. UAV used in experiments is the $\mu$Morus which can be found in the \textit{mmuav\_gazebo} repository \cite{gitLink}, along with its model parameters. Two experiments will be conducted with UAVs using two different methods of CoG variation: MMC in the first case and payload carried by manipulators in the second case. \\
Control parameters for the first case are chosen as follows:
\begin{equation*}
	\text{k}_x = 
	\begin{bmatrix}
		10 &  0  &  0 \\
		 0 & 10  &	0 \\ 
		 0 &  0  & 50 	
	\end{bmatrix}
	\, , \,	
	\text{k}_v =
	\begin{bmatrix}
		3.75 & 0 & 0 \\
		0 & 3.75 & 0 \\
		0 & 0 & 20
	\end{bmatrix}
\end{equation*}
\begin{equation*}
	\text{k}_R = 
	\begin{bmatrix}
		1.5 & 0 & 0 \\
		0 & 1.5 & 0 \\
		0 & 0 & 10
	\end{bmatrix}
	\, , \,
	\text{k}_\Omega = 
	\begin{bmatrix}
		0.65 & 0 & 0 \\
		0 & 0.65 & 0 \\
		0 & 0 & 1.54
	\end{bmatrix}
\end{equation*}

\noindent Rotational control parameters, in the second case, stay the same, while translational parameters are the following: 
\begin{equation*}
	\text{k}_x = 
	\begin{bmatrix}
		7.2 &  0  &  0 \\
		0 & 7.2  &	0 \\ 
		0 &  0  & 50 	
	\end{bmatrix}
	\, , \,	
	\text{k}_v =
	\begin{bmatrix}
		2.6 & 0 & 0 \\
		0 & 2.6 & 0 \\
		0 & 0 & 20
	\end{bmatrix}
\end{equation*}
For both cases, initial parameters are obtained by considering the error dynamics \eqref{error_dynamics_linear} and \eqref{error_dynamics_angular} in the equilibrium state. However, they are further tuned with better position tracking performance in mind.\\
\todo[inline]{how tuned?, lovro: for better position tracking}
It is important to note that the actuator dynamics of moving masses and manipulators is taken in consideration within the Gazebo simulation environment. Furthermore there is a slight transient delay while increasing or decreasing rotor velocity which results in a non-instantaneous control force change. \\
\indent The chosen trajectory tracking problem is formulated as a rotating spiral:
\begin{gather*}
	\textbf{x}_d(t) = [0.4\text{t}; \, 0.5\text{sin}(\pi\text{t}); \, 0.6\text{cos}(\pi\text{t}) + 2] \\
	\textbf{b}_{1,d}(t) = [\text{cos}\left(\frac{\pi}{5}\text{t}\right); \, \text{sin}\left(\frac{\pi}{5}\text{t}\right); \, 0]
\end{gather*}
\begin{figure}[h!]
	\centering
	\includegraphics[width=\columnwidth]{./pictures/both_pos.pdf}
	\caption{Comparison of the desired trajectory $\textbf{x}_d$ and measured position values $\textbf{x}_{mv}$ between both simulation cases. While tracking on x and z axes are reasonably similar, slower position tracking can be observed on the y axis for the second simulation case. Calculated MSE values are 0.0079 and 0.00352 for first and second case respectively.}
	\label{fig:traj_pos}
\end{figure}

\begin{figure}
	\centering
	\begin{minipage}{0.5\columnwidth}
		\centering
		\includegraphics[width=\columnwidth]{./pictures/mmcuav_omega.pdf}
		\caption*{a) UAV with MMC}
		\label{fig:mmcuav_omega}
	\end{minipage}%
	\begin{minipage}{0.5\columnwidth}
		\centering
		\includegraphics[width=\columnwidth]{./pictures/mmuav_omega.pdf}
		\caption*{b) UAV carrying a payload}
		\label{fig:mmuav_omega}
	\end{minipage}
	\caption{Comparison between desired $\mb{\Omega}_d$ and measured $\mb{\Omega}_{mv}$ angular velocity of both simulation cases. Better angular velocity tracking is achieved in the first case, while in the second case a highly oscillatory measured value is observed around x axis.}
\end{figure}

\begin{figure}
	\centering
	\begin{minipage}{0.5\columnwidth}
		\centering
		\includegraphics[width=\columnwidth]{./pictures/mmcuav_control_inputs.pdf}
		\caption*{a) UAV with MMC}
		\label{fig:mmcuav_control}
	\end{minipage}%
	\begin{minipage}{0.5\columnwidth}
		\centering
		\includegraphics[width=\columnwidth]{./pictures/mmuav_control_inputs.pdf}
		\caption*{b) UAV carrying a payload}
		\label{fig:mmuav_control}
	\end{minipage}
	\caption{Figures show control inputs for both simulation cases cases: rotor velocities $\omega_i$, moving mass and payload offsets $r_i$.}
\end{figure}


\begin{figure}[h!]
	\centering
	\includegraphics[width=\columnwidth]{./pictures/both_cog_err.pdf}
	\caption{Comparison between first two components of CoG vector $\textbf{r}_{CoG}$ and between attitude error functions $\Psi$ of both simulation cases. It can be seen that higher magnitude of CoG variation can be achieved using moving masses rather than the carried payload.}
	\label{fig:cog_error}
\end{figure}