\section{Geometric control on SE(3)}

In this section a nonlinear tracking controller will be developed. The main focus will be put on position tracking, therefore the trajectory will consist of a desired position $\textbf{x}_d(t)$ and desired heading $\textbf{b}_{1,d}$ of the body-fixed frame. Since the given position is known ahead of time, one is able to calculate both desired linear velocity $\textbf{v}_d(t)$ and acceleration $\textbf{a}_d(t)$ which will also inherently be included as inputs. \\
The controller will be developed on the nonlinear Lie group SE(3) whose subgroups are the rotation group SO(3) and translation group T(3). The main advantage of using the SO(3) rotation group is to avoid any singularities or ambiguities that may arise when representing rotations with Euler angles or quaternions. \\
Firstly, chosen position and orientation errors will be presented which will also lie on the SE(3) manifold and its tangent space. Using previously defined errors, nonlinear control terms can be chosen. Finally, the stability conditions of the tracking errors will be presented.

\subsection{Tracking errors}

Compatible attitude error function and transport map between tangent bundles of SO(3) are chosen as suggested in \cite{bulloBook} and confirmed in research regarding geometric control with aerial vehicles \cite{LeeClanak1}, \cite{LeeClanak2}, \cite{LeeClanak3}, \cite{LeeClanak4}. Attitude error function on SO(3) along with its compatible transport map are chosen as:
\begin{gather}
	\Psi(\text{R}, \text{R}_d) = \frac{1}{2}tr[I - \text{R}_d^T\text{R}] \label{attErr} \\ 
	\Ta(\text{R}, \text{R}_d) = \text{R}^T \text{R}_d \label{transportMap}
\end{gather}

\noindent Linear position and velocity tracking errors are defined as follows:
\begin{gather}
	\textbf{e}_x = \textbf{x} - \textbf{x}_d \\
	\textbf{e}_v = \textbf{v} - \textbf{v}_d \label{linear_error}
\end{gather}

\noindent Defining attitude and angular velocity tracking errors is not as straight-forward. It is shown in \cite{bulloBook} that the attitude tracking error should be chosen as a left-differential of the attitude error function $\Psi(\text{R}, \text{R}_d)$. It is chosen as follows:
\begin{equation}
	\textbf{e}_\text{R} = \frac{1}{2}(\text{R}_d^T\text{R} - \text{R}^T\text{R}_d)^\text{V}
\end{equation}

\noindent Due to the fact that angular velocities $\mb{\Omega} \in T_\text{R}SO(3)$ and $\mb{\Omega}_d \in T_{\text{R}_d}SO(3)$ lie in different tangential bundles, the proposed left transport map \ref{transportMap} needs to be applied when calculating the tracking error:
\begin{equation}
	\textbf{e}_\Omega = \mb{\Omega} - \text{R}^T\text{R}_d\mb{\Omega}_d \label{angular_error}
\end{equation}

\subsection{Control terms}

Taking in consideration the proposed system dynamics \ref{model2} and \ref{model4}, the force and moment control terms are chosen as follows:

\begin{align}
	\begin{split}
		\textbf{A} = (-& \text{k}_x \textbf{e}_x - \text{k}_v \textbf{e}_v \\
		+& mg\textbf{e}_3 + m\ddot{\textbf{x}}_d \\
		-& m\text{R}\textbf{r}_{CoG}  \times \dot{\mb{\Omega}} - m\text{R}\hat{\mb{\Omega}}\hat{\textbf{r}}_{CoG}\mb{\Omega} ) \\
		f =& \textbf{A} \cdot \text{R}\textbf{e}_3 \label{force_control}
	\end{split}
\end{align}

\begin{align}
	\begin{split}
		\textbf{M} = -& \text{k}_\text{R} \textbf{e}_\text{R} - \text{k}_\Omega \textbf{e}_\Omega \\
			+& \mb{\Omega} \times J\mb{\Omega} - J(\hat{\mb{\Omega}}\text{R}^T\text{R}_d\mb{\Omega}_d - \text{R}^T\text{R}_d\dot{\mb{\Omega}}_d) \\
			+& m\textbf{r}_{CoG} \times \text{R}^T \ddot{\textbf{x}}  \label{moment_control}
	\end{split}
\end{align}
\noindent When error dynamics will be presented, it can be seen that the control terms are chosen in order to negate the undesirable system dynamics.

Desired rotation matrix is constructed in the traditional way when considering geometric control of aerial vehicles \cite{LeeClanak4}, \cite{LeeClanak3}, \cite{LeeClanak2}. The proposed desired rotation matrix is constructed as $\text{R}_d = [\textbf{b}_{1,c}, \textbf{b}_{3,d} \times \textbf{b}_{1,c}, \textbf{b}_{3,d}]$ where component vectors of $\text{R}_d$ are calculated in the following way:
\begin{gather}
	\textbf{b}_{3,d} = \frac{\textbf{A}}{|| \textbf{A} ||} \\
	\textbf{b}_{1,c} = -\frac{(\textbf{b}_{3,d} \times (\textbf{b}_{3,d} \times \textbf{b}_{1,d}))}{||\textbf{b}_{3,d} \times \textbf{b}_{1,d}||}
\end{gather}
\noindent It is also assumed that:
\begin{equation}
	|| \textbf{A} || \neq 0 \label{condition1}
\end{equation}

\noindent The chosen constraint for the trajectory tracking problem differs slightly from the one proposed in \cite{LeeClanak4} due to the fact that different model dynamics are considered in this paper. New trajectory constraints are presented as follows:
\begin{equation}
	|| mge_3 + m\ddot{\textbf{x}}_d 
	- m\text{R}\textbf{r}_{CoG}  \times \dot{\mb{\Omega}} - m\text{R}\hat{\mb{\Omega}}\hat{\textbf{r}}_{CoG}\mb{\Omega}|| < B \label{condition2}
\end{equation}
where B is some positive constant. 

Desired angular velocity and acceleration also need to be considered in this trajectory tracking problem. One is able to calculate the desired angular velocity and acceleration using $\text{R}_d$ and its derivatives in the following way:
\begin{gather}
	\hat{\mb{\Omega}}_d = \text{R}_d^T \dot{\text{R}}_d \\
	\dot{\hat{\mb{\Omega}}}_d = - \hat{\mb{\Omega}}_d\hat{\mb{\Omega}}_d + \text{R}_d^T \ddot{\text{R}}_d
\end{gather}

Derivatives of $\text{R}_d$ are easily calculated using the backwards differentiation method. It has to be noted that due to stability issues, computation rate of desired angular velocity and acceleration has to be lesser than the overall simulation rate. For further implementation details, please refer to \cite{gitLink}.

\subsection{Error dynamics and stability discussion}

In this section linear and angular error dynamics will be presented. First of all, derivatives over time need to be calculated for linear \ref{linear_error} and angular \ref{angular_error} tracking errors:
\begin{gather}
	\textbf{e}_v = \ddot{\textbf{x}} - \ddot{\textbf{x}}_d \label{linear_error_dynamics}\\
	\textbf{e}_\Omega = \dot{\mb{\Omega}} + \hat{\mb{\Omega}}\text{R}^T\text{R}_d\mb{\Omega}_d - \text{R}^T\text{R}_d\dot{\mb{\Omega}}_d \label{angular_error_dynamics}
\end{gather}

\noindent After including \ref{model2} and \ref{model4} in \ref{linear_error_dynamics} and \ref{angular_error_dynamics} respectively, the following equations are obtained:
\begin{align}
	\label{lin_dynamics_full}
	\begin{split}
		m\textbf{e}_v = & - mg\textbf{e}_3 - m\ddot{\textbf{x}}_d \\
			&+ m\text{R}\textbf{r}_{CoG}  \times \dot{\mb{\Omega}} + m\text{R}\hat{\mb{\Omega}}\hat{\textbf{r}}_{CoG}\mb{\Omega} \\
			&+ \textbf{A} + \textbf{X}	
	\end{split} \\
	\label{ang_dynamics_full}
	\begin{split}
		J\textbf{e}_\Omega = &\textbf{M} - \mb{\Omega} \times J\mb{\Omega} \\
			&+ J(\hat{\mb{\Omega}}\text{R}^T\text{R}_d\mb{\Omega}_d - \text{R}^T\text{R}_d\dot{\mb{\Omega}}_d) \\
			&+ m\textbf{r}_{CoG} \times \text{R}^T \ddot{\textbf{x}}
	\end{split}
\end{align}

\noindent Note that in \ref{lin_dynamics_full} $\textbf{A}\in \mathbb{\text{R}}^3$ is regarded as a control force for the translational dynamics, mentioned in \ref{force_control}, while $\textbf{X}\in\mathbb{\text{R}}^3$ is a bounded term that arises when deriving this equation which equals:
\begin{equation}
	\textbf{X} = \frac{f}{(\text{R}_d\textbf{e}_3)^T\text{R}\textbf{e}_3}(\text{R}_d \textbf{e}_3 - ((\text{R}_d\textbf{e}_3)^T\text{R}\textbf{e}_3)\text{R}\textbf{e}_3)
\end{equation}

\noindent After substituting control force from \ref{force_control} and \ref{moment_control} in \ref{lin_dynamics_full} and \ref{ang_dynamics_full} respectively the final form of error dynamics is obtained:
\begin{gather}
	m\textbf{e}_v = -k_x \textbf{e}_x - k_v \textbf{e}_v + \textbf{X} \label{error_dynamics_linear}\\ 
	J\textbf{e}_\Omega = -k_\text{R} \textbf{e}_\text{R} - \text{k}_\Omega \textbf{e}_\Omega \label{error_dynamics_angular}
\end{gather}

Having started with a different mathematical model of the UAV than the previous research done on this subject, applying the newly formed control terms \ref{force_control} and \ref{moment_control} and taking in consideration initial assumptions \ref{condition1} and \ref{condition2} one is able to derive identical translational and rotational error dynamics as found in \cite{LeeClanak1}, \cite{LeeClanak4}. \\
Therefore, to avoid redundancy, the full stability proof will not be presented in this paper. However, the final conclusions for exponential asymptotic stability of the attitude error function and attraction to the zero-equilibrium state of tracking errors will be outlined. \\

If the initial UAV configuration satisfies the following conditions:
\begin{gather}
	\Psi (\text{R}(0), \text{R}_d(0)) < 2 \\
	||\textbf{e}_\Omega(0)||^2 < \frac{2}{\lambda_{min}(J)}k_\text{R}(2 - \Psi(\text{R}(0), \text{R}_d(0))
\end{gather}
it can be shown that tracking errors of the whole system will reach zero-equilibrium state and the attitude function will be exponentially bounded as:
\begin{equation}
	\Psi(\text{R}(t), \text{R}_d(t)) \leq \text{min}\{2, \alpha e^{-\beta t} \}
\end{equation}
for some positive constants $\alpha$ and $\beta$.
