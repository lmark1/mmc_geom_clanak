\section{Geometric control on SE(3)} \label{sec:control}

The main focus of the proposed controller is put on position tracking. Therefore, the trajectory consists of a desired position $\textbf{x}_d(t)$ and a desired heading $\textbf{b}_{1,d}(t)$ of the body-fixed frame. Since the given position is known ahead of time, one is able to calculate both desired linear velocity $\textbf{v}_d(t)$ and acceleration $\textbf{a}_d(t)$ which are also inherently included as inputs. \\
The controller is developed on the nonlinear Lie group SE(3) consisting of the rotation group SO(3) and translation group T(3). It is cascade in structure, as seen in \ref{fig:control_scheme} with position tracking block following attitude tracking block. The main advantage of using the SO(3) rotation group is to avoid any singularities or ambiguities that may arise when representing rotations with Euler angles or quaternions. \\
Controller synthesis is carried out as follows. First, position and orientation tracking errors are presented as the proportional and derivative part of the controller. In addition, nonlinear control terms are chosen to compensate proposed model dynamics. Finally, exponential stability conditions for the initial UAV configuration are shown.

\begin{figure}
	\centering
	\includegraphics[width=\columnwidth]{./pictures/ControlScheme.png}	
	\caption{Control scheme for the UAV carrying a payload. When considering UAV with MMC \textit{Inverse Jacobian} block becomes the identity matrix. This means that the control inputs $d_x$ and $d_y$ are directly sent as system inputs.}
	\label{fig:control_scheme}
\end{figure}

\subsection{Tracking errors}

Compatible attitude error function and transport map between tangent bundles of SO(3) are chosen as suggested in \cite{bulloBook} and confirmed in research regarding geometric control with aerial vehicles \cite{LeeClanak1}, \cite{LeeClanak2}, \cite{LeeClanak3}, \cite{LeeClanak4}. Attitude error function on SO(3) $	\Psi(\text{R}, \text{R}_d)$  along with its compatible transport map $\Ta(\text{R}, \text{R}_d)$ are chosen as:
\begin{gather}
	\Psi(\text{R}, \text{R}_d) = \frac{1}{2} \, \text{tr} [\, \text{I} - \text{R}_d^T\text{R} \,] \label{attErr} \\ 
	\Ta(\text{R}, \text{R}_d) = \text{R}^T \text{R}_d \label{transportMap}
\end{gather}
Linear position and velocity tracking errors are defined as follows:
\begin{gather}
	\textbf{e}_x = \textbf{x} - \textbf{x}_d \\
	\textbf{e}_v = \textbf{v} - \textbf{v}_d \label{linear_error}
\end{gather}
It is shown in \cite{bulloBook} that the attitude tracking error should be chosen as a left-differential of the attitude error function $\Psi(\text{R}, \text{R}_d)$ as follows:
\begin{equation}
	\textbf{e}_\text{R} = \frac{1}{2} (\, \text{R}_d^T\text{R} - \text{R}^T\text{R}_d \,)^\vee
\end{equation}
Due to the fact that angular velocities $\mb{\Omega} \in T_\text{R}SO(3)$ and $\mb{\Omega}_d \in T_{\text{R}_d}SO(3)$ evolve in different tangential bundles, the proposed left transport map \eqref{transportMap} needs to be applied when calculating the angular velocity tracking error:
\begin{equation}
	\textbf{e}_\Omega = \mb{\Omega} - \text{R}^T\text{R}_d\mb{\Omega}_d \label{angular_error}
\end{equation}

\subsection{Control terms}
After defining all tracking errors, one can start constructing the control terms. Taking in consideration the proposed system dynamics \eqref{model2} and \eqref{model4}, the force and moment control terms are chosen as follows:
\begin{align}
	\begin{split}
		\textbf{A} =& (- \text{k}_x \textbf{e}_x - \text{k}_v \textbf{e}_v + m\ddot{\textbf{x}}_d \\
		& + mg\textbf{e}_3 - m_t \text{R} \,(\, \textbf{r}_{CoG} \times \dot{\mb{\Omega}} \,) \\
		& - m_t \text{R} \,[\, \mb{\Omega} \times (\textbf{r}_{CoG} \times \mb{\Omega}) \,]\, ) \\
		f =& \textbf{A} \cdot \text{R}\textbf{e}_3 \label{force_control}
	\end{split}
\end{align}
\begin{align}
	\begin{split}
		\textbf{M} = -& \text{k}_\text{R} \textbf{e}_\text{R} - \text{k}_\Omega \textbf{e}_\Omega \\
			- & \text{J} \,(\, \reallywidehat{\mb{\Omega}}\text{R}^T\text{R}_d\mb{\Omega}_d - \text{R}^T\text{R}_d\dot{\mb{\Omega}}_d \,) \\
			+ & \mb{\Omega} \times \text{J}\mb{\Omega} + m\textbf{r}_{CoG} \times \text{R}^T \ddot{\textbf{x}}  \label{moment_control} 
	\end{split}
\end{align}
\indent Desired rotation matrix is constructed in the conventional way when considering geometric control of aerial vehicles \cite{LeeClanak4}, \cite{LeeClanak3}, \cite{LeeClanak2}. The proposed desired rotation matrix is constructed as $\text{R}_d = [\textbf{b}_{1,c}, \textbf{b}_{3,d} \times \textbf{b}_{1,c}, \textbf{b}_{3,d}]$ where component vectors of $\text{R}_d$ are calculated in the following way:
\begin{gather}
	\textbf{b}_{3,d} = \frac{\textbf{A}}{|| \textbf{A} ||} \label{eqn:desired_thrust}\\
	\textbf{b}_{1,c} = -\frac{(\, \textbf{b}_{3,d} \times (\, \textbf{b}_{3,d} \times \textbf{b}_{1,d} \,)\, )}{||\textbf{b}_{3,d} \times \textbf{b}_{1,d}||}
\end{gather}
The chosen constraint for the trajectory tracking problem differ slightly from the one proposed in \cite{LeeClanak4}. Due to the fact that model dynamics which include variable CoG are considered in this paper, new trajectory constraints are presented as follows:
\begin{equation}
	|| m_tge_3 + m\ddot{\textbf{x}}_d 
	- m_t\text{R} (\textbf{r}_{CoG}  \times \dot{\mb{\Omega}}) - m_t \text{R}[\mb{\Omega} \times (\textbf{r}_{CoG} \times \mb{\Omega})]|| < B \, , \label{condition2}
\end{equation}
where B is some positive constant. \\
\indent Desired angular velocity and acceleration also need to be considered in this trajectory tracking problem. One is able to calculate the desired angular velocity and acceleration using $\text{R}_d$ and its derivatives in the following way:
\begin{gather}
	\reallywidehat{\mb{\Omega}}_d = \text{R}_d^T \dot{\text{R}}_d \label{eqn:omega_d}\\
	\dot{\reallywidehat{\mb{\Omega}}}_d = - \reallywidehat{\mb{\Omega}}_d\reallywidehat{\mb{\Omega}}_d + \text{R}_d^T \ddot{\text{R}}_d \label{eqn:alpha_d}
\end{gather}
Derivatives of $\text{R}_d$ are calculated using the backwards differentiation method. Another important note is that the computation rate of the desired angular velocity and acceleration is slower than the overall simulation rate. For further implementation details, please refer to \cite{gitLink}.

\subsection{Error dynamics} \label{ssec:error_dynamics}

In order to express error dynamics, one needs to calculate the time derivatives of linear \eqref{linear_error} and angular \eqref{angular_error} tracking errors:
\begin{gather}
	\textbf{e}_v = \dot{\textbf{x}} - \dot{\textbf{x}}_d \label{linear_error_dynamics}\\
	\textbf{e}_\Omega = \dot{\mb{\Omega}} + \reallywidehat{\mb{\Omega}}\text{R}^T\text{R}_d\mb{\Omega}_d - \text{R}^T\text{R}_d\dot{\mb{\Omega}}_d \label{angular_error_dynamics}
\end{gather}
After including \eqref{model2} and \eqref{model4} in \eqref{linear_error_dynamics} and \eqref{angular_error_dynamics} respectively, the following equations are obtained:
\begin{align}
	\label{lin_dynamics_full}
	\begin{split}
		m\textbf{e}_v = & - mg\textbf{e}_3 - m\ddot{\textbf{x}}_d \\
			& + m_t \text{R} \,(\, \textbf{r}_{CoG}  \times \dot{\mb{\Omega}} \,) \\
			& + m_t \text{R} \,[\, \mb{\Omega} \times (\, \textbf{r}_{CoG} \times \mb{\Omega} \,)\, ] \\
			&+ \textbf{A} + \textbf{X}	
	\end{split} \\
	\label{ang_dynamics_full}
	\begin{split}
		\text{J}\textbf{e}_\Omega =\,  &\text{J} \,(\, \reallywidehat{\mb{\Omega}}\text{R}^T\text{R}_d\mb{\Omega}_d - \text{R}^T\text{R}_d\dot{\mb{\Omega}}_d \,) \\
			& - \mb{\Omega} \times \text{J}\mb{\Omega} + m_t \textbf{r}_{CoG} \times \text{R}^T \ddot{\textbf{x}} \\
			& + \textbf{ M}
	\end{split}
\end{align}
Note that in \eqref{lin_dynamics_full} $\textbf{X}\in\mathbb{\text{R}}^3$ is a bounded term which equals:
\begin{equation}
	\textbf{X} = \frac{f}{(\text{R}_d\textbf{e}_3)^T\text{R}\textbf{e}_3} \,(\, \text{R}_d \textbf{e}_3 - (\, (\text{R}_d\textbf{e}_3)^T\text{R}\textbf{e}_3 \,)\, \text{R}\textbf{e}_3 \,)
\end{equation}
After substituting control force from \eqref{force_control} and \eqref{moment_control} in \eqref{lin_dynamics_full} and \eqref{ang_dynamics_full} respectively the final form of error dynamics is obtained:
\begin{gather}
	m\textbf{e}_v = -k_x \textbf{e}_x - k_v \textbf{e}_v + \textbf{X} \label{error_dynamics_linear}\\ 
	\text{J}\textbf{e}_\Omega = -k_\text{R} \textbf{e}_\text{R} - \text{k}_\Omega \textbf{e}_\Omega \label{error_dynamics_angular}
\end{gather}

\subsection{Stability discussion}

We begin by observing \eqref{error_dynamics_linear} and \eqref{error_dynamics_angular}. 
It has to be taken in account that the derived error dynamics are identical to those presented in \cite{LeeClanak1} and \cite{LeeClanak4}, even though the proposed model and control terms in this paper differ from previous research. \\
Therefore, to avoid redundancy, the complete stability proof is omitted for brevity. Instead, only final conclusions for attitude error function $\Psi (\text{R}(t), \text{R}_d(t))$ exponential asymptotic stability and tracking error attraction to the zero-equilibrium state are outlined as established in \cite{LeeClanak1}.

Granted the initial UAV configuration satisfies the following conditions:
\begin{gather}
	\Psi (\text{R}(0), \text{R}_d(0)) < 2 \\
	||\textbf{e}_\Omega(0)||^2 < \frac{2}{\lambda_{min}(\text{J})}k_\text{R}(2 - \Psi(\text{R}(0), \text{R}_d(0)) \, ,
\end{gather}
it can be shown that tracking errors of the whole system reaches zero-equilibrium state and the attitude function is exponentially bounded as:
\begin{equation}
	\Psi(\text{R}(t), \text{R}_d(t)) \leq \text{min}\{2, \alpha e^{-\beta t} \}
\end{equation}
for some positive constants $\alpha$ and $\beta$.